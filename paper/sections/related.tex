\section{Related Work}\label{sec:related}

\subsection{Identifcation and Counting of Hosts Behind NAT Using Machine
Learning \cite{shukla2022identification}}
In this, we use ICS data set to train our model which contains traffic flow from 1116 hosts. We identify each host with
the help of source IP address and mark them as a unique
class. Once classes are assigned to each fow, we train our
model with the help of ICS data set. After training our data
model, we have used diferent data sets which contains NATted trafc flow from multiple hosts to test our trained model.
Our trained model then classifes each fow to a class with
the help of all the selected features. Unique classes are then
counted to count number of hosts.

How long the duraiong of ICS dataset? is 400MB enough? It had 460K flows, not natted

The second dataset, MTA: 143 hosts, natted, and non natted
The third dataset, their own, 73 PC hosts 6 hours, natted and non natted, only web, running a script 
The test 

The paper relies heavily on reference 17 for choice of flow analayzer, choice of filtering method 

Much different featurs than us, we are getting mostly the inter-arrival time and flow based stats while the paper gets mostliy thibngs related to window size and ACks. 
ALso, they only not consider src port, but destination port is considered. 

Do they consider UDP packets? Their most important features are tcp

Why not considering all the features and then eliminating shortcuts?
They also had to reduce the number of features and it improved the preformance --> maybe they did not have large enough dataset --> their dataset size is equal to number of flows but ours can be as big is n chose 2 (half of n squared): They use multiclass calssifier 

It redifines some ML stuff to just fill out the paper 

They got 98 percent accuracy when test and train performed on ICS dataset, it is a shortcut though, accuracy dropped to 89 on lab dataset 



IMPORTANT: Tranalayser considers the IAT 

\subsection{Identifying NAT Devices to Detect Shadow IT: A Machine Learning Approach \cite{nassar2021identifying}}




\subsection{A Generalizable Machine Learning Model for NAT
Detection \cite{nassar2023generalizable}}
Based on their previous work, it is just detecting the existences of NAT, not complete set of features, users transfer learning, not a good dataset 



\subsection{Exploring nat detection and host identification using machine learning \cite{khatouni2019exploring}}

check this paper: Integrating machine learning
with off-the-shelf traffic flow features for http/https traffic classification,

Also based on this paper, tranalyzer had the best preformance 


\subsection{How Polynomial Regression Improves DeNATing \cite{adler2023polynomial}}
The refrences one to 9: check them all

The main ID of using IPID sucks: One is because of concurrent connections, it weould not get increamented but it would increase a lot in time, the other thing is DNS resovler in the network behind NAT and also DNS caching 

It is TCP/IP based, what about UDP 

Identifying packets belong to the same flow using TCP timestamp: Isn't it already solved by 4 tuples using pcapsplitter or cicflow?


Assumption about IP IP assignmet by increameing it for each packet either per flow or globally

They use TTL that can be a shortcut, but what about different devices use differnt Initial TTL: This can be a good info 

Check the TCP window Scale size 

OS fingerprinting using TCP/IP header fileds 

Communication with speficif domains for OS fingerprinting (ref 15)

check netflow records 

Check Timestamp filed and the two studies based on timestamp

In old versions of Windows OS, the IP-ID field was
implemented as a simple counter. Newer versions use separate
counters per destination address. IOS assigns a random number
to IP-ID, and some versions of Linux always set it to zero.

The IP-ID method requires long time of catpuring data cause for creating sequence I guess they need more than several DNS requests. Also, how precies is that?

